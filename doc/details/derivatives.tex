\documentclass{article}
\usepackage{amsmath,amssymb}
\usepackage{bm}

\begin{document}

\section*{Derivative of the Anisotropic Covariance Function (Matérn) in 2D and 3D}

We consider an anisotropic covariance function of the form:
\[
k(h) = C\left( r(h) \right), \quad \text{where } r(h) = \sqrt{h^\top T h}
\]
with:
\begin{itemize}
  \item \( h \in \mathbb{R}^d \) is the spatial displacement vector,
  \item \( T = R^\top \Lambda R \in \mathbb{R}^{d \times d} \) is a symmetric positive definite anisotropy tensor,
  \item \( \Lambda = \mathrm{diag}(\lambda_1, \lambda_2, [\lambda_3]) \), with \( \lambda_i = \frac{1}{s_i^2} \),
  \item \( R \in SO(d) \) is a rotation matrix.
\end{itemize}

We define the rotated coordinate \( \tilde{h} = R h \), so:
\[
r(h)^2 = \tilde{h}^\top \Lambda \tilde{h}
\]

\section*{2D Case}

Let \( d = 2 \), and \( R(\theta) \) be a rotation by angle \( \theta \):
\[
R(\theta) = \begin{pmatrix}
\cos\theta & -\sin\theta \\
\sin\theta & \cos\theta
\end{pmatrix}
\]

Let \( J = \begin{pmatrix} 0 & -1 \\ 1 & 0 \end{pmatrix} \) be the generator of infinitesimal rotations.

\subsection*{Derivative with respect to scale \( s_i \)}

Since \( \lambda_i = 1/s_i^2 \), we obtain:
\[
\frac{\partial k(h)}{\partial s_i} = - \frac{C'(r)}{r} \cdot \frac{\tilde{h}_i^2}{s_i^3}
\]

\subsection*{Derivative with respect to angle \( \theta \)}

\[
\frac{\partial k(h)}{\partial \theta} = \frac{C'(r)}{r} \cdot \tilde{h}^\top \Lambda J \tilde{h}
\]

\subsection*{Behavior at \( h = 0 \)}

In both cases, the derivative tends to 0 as \( h \to 0 \), so:
\[
\left. \frac{\partial k(h)}{\partial \xi} \right|_{h = 0} = 0
\quad \text{for } \xi \in \{s_1, s_2, \theta\}
\]

\section*{3D Case}

Let \( d = 3 \), and \( R \in SO(3) \) be a rotation matrix parameterized by Euler angles \( (\theta_1, \theta_2, \theta_3) \) (e.g., ZYX convention: yaw, pitch, roll).

\subsection*{Derivative with respect to scale \( s_i \)}

Same as in 2D:
\[
\frac{\partial k(h)}{\partial s_i} = - \frac{C'(r)}{r} \cdot \frac{\tilde{h}_i^2}{s_i^3}
\]

\subsection*{Derivative with respect to orientation parameters}

Let \( J_k \in \mathfrak{so}(3) \) be the skew-symmetric generator for infinitesimal rotation around axis \( k \). Then:
\[
\frac{\partial k(h)}{\partial \theta_k} = \frac{C'(r)}{r} \cdot \tilde{h}^\top \Lambda J_k \tilde{h}
\]

\subsection*{Infinitesimal rotation matrices}

The generators \( J_1, J_2, J_3 \in \mathbb{R}^{3 \times 3} \) are:

\[
J_1 = \begin{pmatrix}
0 & 0 & 0 \\
0 & 0 & -1 \\
0 & 1 & 0
\end{pmatrix}, \quad
J_2 = \begin{pmatrix}
0 & 0 & 1 \\
0 & 0 & 0 \\
-1 & 0 & 0
\end{pmatrix}, \quad
J_3 = \begin{pmatrix}
0 & -1 & 0 \\
1 & 0 & 0 \\
0 & 0 & 0
\end{pmatrix}
\]

Each corresponds to rotation around the \( x \)-, \( y \)-, and \( z \)-axes, respectively.

\subsection*{Note on parameterization}

The matrices \( J_k \) apply directly for Lie algebra–based parametrization. If using Euler angles, chain rule must be applied:
\[
\frac{\partial k}{\partial \theta_{\text{Euler}}} = \sum_{k=1}^3 \frac{\partial k}{\partial \theta_k} \cdot \frac{\partial \theta_k}{\partial \theta_{\text{Euler}}}
\]


\section{Profiled log-likelihood}

We want to maximize the log-likelihood function:

\[\ell(\theta,\beta) = \frac{1}{2} \log |Q_\theta| -\frac{1}{2}(y-X\beta)'Q_\theta(y-X\beta) \]

Let's denote

\[\hat\beta(\theta) = (X'Q_\theta X)^{-1}X'Q_\theta y\]
We can easily show that  \[\hat\beta(\theta) = \arg\max_\beta \ell(\theta,\beta)\]

So we can define the profiled log-likelihood as:
\[p\ell(\theta) = \ell(\theta,\hat\beta(\theta)) = \frac{1}{2} \log |Q_\theta| -\frac{1}{2}(y-X\hat\beta(\theta))'Q_\theta(y-X\hat\beta(\theta)) \]

We can rewrite it as:

\[p\ell(\theta) = \frac{1}{2} \log |Q_\theta| -\frac{1}{2}\hat r(\theta)'Q_\theta\hat r(\theta) 
\]
where \(\hat r(\theta) = y-X\hat\beta(\theta)\).

Let's denote \( \hat y(\theta) = X\hat\beta(\theta) \).

The derivative of the profiled log-likelihood with respect to \(\theta\) is given by:
\[\frac{\partial p\ell(\theta)}{\partial \theta} = \frac{1}{2} \mathrm{tr}\left(Q_\theta^{-1} \frac{\partial Q_\theta}{\partial \theta}\right) 
 -  \hat r(\theta)^\top Q_\theta \frac{\partial \hat r(\theta)}{\partial \theta}
 -  \frac{1}{2}\hat r(\theta)^\top  \frac{\partial Q_\theta}{\partial \theta} \hat r(\theta) \]     
 
 Now let's compute the derivatives of \(\hat r(\theta)\):

\[\hat r(\theta) = y - X\hat\beta(\theta) = y - X(X'Q_\theta X)^{-1}X'Q_\theta y\]
The derivative of \(\hat r(\theta)\) with respect to \(\theta\) is:

\begin{align}
\frac{\partial \hat r(\theta)}{\partial \theta} &= -X\frac{\partial \hat\beta(\theta)}{\partial \theta} 
\end{align}

with 

\begin{align}
  \frac{\partial \hat \beta(\theta)}{\partial \theta} &= -(X'Q_\theta X)^{-1}X'\frac{\partial Q_\theta}{\partial \theta}X (X'Q_\theta X)^{-1} X'Q_\theta y + (X'Q_\theta X)^{-1}X' \frac{\partial Q_\theta}{\partial \theta}y\\
&=(X'Q_\theta X)^{-1}X'\frac{\partial Q_\theta}{\partial \theta} \hat r(\theta)
\end{align}

So 

\[\frac{\partial \hat r(\theta)}{\partial \theta} = -X(X'Q_\theta X)^{-1}X'\frac{\partial Q_\theta}{\partial \theta} \hat r(\theta)\]
And we can compute

\begin{align*}
  \hat r(\theta)^\top Q_\theta \frac{\partial \hat r(\theta)}{\partial \theta} &= - \hat r(\theta)^\top Q_\theta X(X'Q_\theta X)^{-1}X'\frac{\partial Q_\theta}{\partial \theta} \hat r(\theta)\\
&= - (y-\hat y(\theta))^\top Q_\theta X(X'Q_\theta X)^{-1}X'\frac{\partial Q_\theta}{\partial \theta} \hat r(\theta)\\
&= - y^\top Q_\theta X(X'Q_\theta X)^{-1}X'\frac{\partial Q_\theta}{\partial \theta} \hat r(\theta)+ \hat y(\theta)^\top Q_\theta X(X'Q_\theta X)^{-1}X'\frac{\partial Q_\theta}{\partial \theta} \hat r(\theta)\\
&= - y^\top Q_\theta X(X'Q_\theta X)^{-1}X'\frac{\partial Q_\theta}{\partial \theta} \hat r(\theta) + y^\top Q_\theta X(X'Q_\theta X)^{-1}X' Q_\theta X(X'Q_\theta X)^{-1}X'\frac{\partial Q_\theta}{\partial \theta} \hat r(\theta)\\
&=  - y^\top Q_\theta X(X'Q_\theta X)^{-1}X'\frac{\partial Q_\theta}{\partial \theta} \hat r(\theta)+ y^\top Q_\theta X(X'Q_\theta X)^{-1}X'\frac{\partial Q_\theta}{\partial \theta} \hat r(\theta)\\
&= 0
\end{align*}



Finally, we can rewrite the derivative of the profiled log-likelihood as:
\[\frac{\partial p\ell(\theta)}{\partial \theta} = \frac{1}{2} \mathrm{tr}\left(Q_\theta^{-1} \frac{\partial Q_\theta}{\partial \theta}\right) 
 -  \frac{1}{2}\hat r(\theta)^\top  \frac{\partial Q_\theta}{\partial \theta} \hat r(\theta) \]


 For the second derivative, we have:

 \begin{align*}
\frac{\partial^2 p\ell}{\partial \theta_i\partial\theta_j}(\theta) &= -\frac{1}{2} \mathrm{tr}\left(Q_\theta^{-1} \frac{\partial Q_\theta}{\partial \theta_i} Q_\theta^{-1} \frac{\partial Q_\theta}{\partial \theta_j}\right) +
\frac{1}{2} \mathrm{tr}\left(Q_\theta^{-1} \frac{\partial^2 Q_\theta}{\partial \theta_i\partial\theta_j}\right)-  
\frac{1}{2}\hat r(\theta)^\top  \frac{\partial^2 Q_\theta}{\partial \theta_i\partial\theta_j} \hat r(\theta)\\
&-\hat r(\theta)^\top\frac{\partial Q_\theta}{\partial \theta_i} \frac{\partial\hat r(\theta)}{\partial \theta_j}\\
&= -\frac{1}{2} \mathrm{tr}\left(Q_\theta^{-1} \frac{\partial Q_\theta}{\partial \theta_i} Q_\theta^{-1} \frac{\partial Q_\theta}{\partial \theta_j}\right) +
\frac{1}{2} \mathrm{tr}\left(Q_\theta^{-1} \frac{\partial^2 Q_\theta}{\partial \theta_i\partial\theta_j}\right)-  
\frac{1}{2}\hat r(\theta)^\top  \frac{\partial^2 Q_\theta}{\partial \theta_i\partial\theta_j} \hat r(\theta)\\
&+\hat r(\theta)^\top\frac{\partial Q_\theta}{\partial \theta_i}X(X'Q_\theta X)^{-1}X'\frac{\partial Q_\theta}{\partial \theta_j} \hat r(\theta)\\
 \end{align*}

We can deduce the terms of the "Fisher information matrix" by taking the expectation of the second derivative of the profiled log-likelihood:

\begin{align*}
  E\left[\frac{\partial^2 p\ell}{\partial \theta_i\partial\theta_j}(\theta)\right] &=-\frac{1}{2} \mathrm{tr}\left(Q_\theta^{-1} \frac{\partial Q_\theta}{\partial \theta_i} Q_\theta^{-1} \frac{\partial Q_\theta}{\partial \theta_j}\right) +
  \frac{1}{2} \mathrm{tr}\left(Q_\theta^{-1} \frac{\partial^2 Q_\theta}{\partial \theta_i\partial\theta_j}\right)-  
  \frac{1}{2}\mathrm{tr}\left(  \frac{\partial^2 Q_\theta}{\partial \theta_i\partial\theta_j} E\left[\hat r(\theta)\hat r(\theta)^\top\right]\right)\\
  &+\mathrm{tr}\left( \frac{\partial Q_\theta}{\partial \theta_i}X(X'Q_\theta X)^{-1}X'\frac{\partial Q_\theta}{\partial \theta_j}E\left[ \hat r(\theta)\hat r(\theta)^\top\right]\right)\\
  &=-\frac{1}{2} \mathrm{tr}\left(Q_\theta^{-1} \frac{\partial Q_\theta}{\partial \theta_i} Q_\theta^{-1} \frac{\partial Q_\theta}{\partial \theta_j}\right) +  
  \frac{1}{2}\mathrm{tr}\left(  \frac{\partial^2 Q_\theta}{\partial \theta_i\partial\theta_j} X(X'Q_\theta X)^{-1}X'\right)\\
  &+\mathrm{tr}\left( \frac{\partial Q_\theta}{\partial \theta_i}X(X'Q_\theta X)^{-1}X'\frac{\partial Q_\theta}{\partial \theta_j}Q_\theta^{-1}\right)\\
  &-\mathrm{tr}\left( \frac{\partial Q_\theta}{\partial \theta_i}X(X'Q_\theta X)^{-1}X'\frac{\partial Q_\theta}{\partial \theta_j}X(X'Q_\theta X)^{-1}X'\right)\\
\end{align*}

\end{document}
